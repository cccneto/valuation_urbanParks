% !TeX program = pdfLaTeX
\documentclass[smallextended]{svjour3}       % onecolumn (second format)
%\documentclass[twocolumn]{svjour3}          % twocolumn
%
\smartqed  % flush right qed marks, e.g. at end of proof
%
\usepackage{amsmath}
\usepackage{graphicx}
\usepackage[utf8]{inputenc}

\usepackage[hyphens]{url} % not crucial - just used below for the URL
\usepackage{hyperref}
\providecommand{\tightlist}{%
  \setlength{\itemsep}{0pt}\setlength{\parskip}{0pt}}

%
% \usepackage{mathptmx}      % use Times fonts if available on your TeX system
%
% insert here the call for the packages your document requires
%\usepackage{latexsym}
% etc.
%
% please place your own definitions here and don't use \def but
% \newcommand{}{}
%
% Insert the name of "your journal" with
% \journalname{myjournal}
%

%% load any required packages here



% Pandoc citation processing


\begin{document}

\title{VALUATION OF CULTURAL ECOSYSTEM SERVICES IN URBAN
PARKS \thanks{Grants or other notes about the article that should go on
the front page should be placed here. General acknowledgments should be
placed at the end of the article.} }
 \subtitle{teremos um subtitulo?} 

    \titlerunning{Short form of title (if too long for head)}

\author{  Carlos Eduardo de Menezes Silva \and  Claudiano Carneiro da
Cruz Neto \and  }

    \authorrunning{ Short form of author list if too long for running
head }

\institute{
        Carlos Eduardo de Menezes Silva \at
     Department of YYY, Federal Institute of Pernambuco \\
     \email{\href{mailto:carlosmenezes@recife.ifpe.edu.br}{\nolinkurl{carlosmenezes@recife.ifpe.edu.br}}}  %  \\
%             \emph{Present address:} of F. Author  %  if needed
    \and
        Claudiano Carneiro da Cruz Neto \at
     Center for Agricultural, Environmental and Biological Sciences,
Federal University of Reconcavo of Bahia \\
     \email{\href{mailto:cneto@ufrb.edu.br}{\nolinkurl{cneto@ufrb.edu.br}}}  %  \\
%             \emph{Present address:} of F. Author  %  if needed
    \and
    }

\date{Received: date / Accepted: date}
% The correct dates will be entered by the editor


\maketitle

\begin{abstract}
The text of your abstract. 150 -- 250 words.
\\
\keywords{
        key \and
        dictionary \and
        word \and
    }

    \subclass{
                    MSC code 1 \and
                    MSC code 2 \and
            }

\end{abstract}


\def\spacingset#1{\renewcommand{\baselinestretch}%
{#1}\small\normalsize} \spacingset{1}


\hypertarget{intro}{%
\section{Introduction}\label{intro}}

The world population may grow 26\% by the year 2050, from the current
7.7 billion to 9.7 billion (UN, 2019). This will lead to an even greater
increase in the population in urban areas, especially in regions such as
Latin America that already have around 81\% of their population living
in these areas (UN, 2018). This scenario increases the concern with
maintaining and / or improving the well-being conditions for this
population. And in this regard, it is considered that the well-being of
the inhabitants of urban areas depends on an adequate supply of
ecosystem services (Keeler et al., 2019).

Understood as nature's contributions to people, ecosystem services are
vital for human existence and good quality of life. However, they are
generally distributed unevenly in space, time and between different
segments of society (IPBES, 2019). Of particular interest to the
well-being of urban populations, cultural ecosystem services are public
goods, produced by ecosystems that affect people's physical and mental
states. Cultural services are characterized mainly as environments,
places or environmental situations that give rise to changes in people's
physical or mental states (Haines-Young \& Potschin, 2018). And
therefore, in a context of greater urbanization, special attention
should be paid to urban green spaces as a guarantee of offering this
type of ecosystem service, and ensuring the integration of these spaces
in city planning (Liu, Remme, Hamel, Nong, \& Ren, 2020).

These benefits provided by urban green areas are still important when
cities seek to offer a variety of services that contribute to increasing
the quality of life of their inhabitants. Initiative that already found
support in the agenda of the Sustainable Development Goals - SDGs, gains
even more momentum with the greater importance given to these spaces
after the long period of confinement caused by the Pandemic of COVID-19.

However, unlike developed countries, in Brazil and Latin America, the
literature on this issue is still scarce. There are few studies that
explore the association between green areas and well-being in urban
areas in Brazil (AMATO-LOURENÇO, MOREIRA, ARANTES, FILHO, \& MAUAD,
2016; Camargo et al., 2018; Londe \& Mendes, 2014; Silveira \& Junger,
2018). There are also few studies on the economic valuation of these
areas. These productions are concentrated in the southern region of the
country and focused on individual parks. And in general, cultural
ecosystem services are less evaluated or evaluated inappropriately
(Ridding et al., 2017) and for this reason they are seldom present in
studies in this area (Boerema, Rebelo, Bodi, Esler, \& Meire, 2017).
This scarcity makes it difficult to understand the potential benefit
that urban green spaces can bring to the Brazilian population (Arana \&
Xavier, 2017). And the importance of investments by the government in
these areas.

In this article, we seek to contribute to fill this information gap and
demonstrate the importance of urban parks as sources of provision of
cultural ecosystem services in the city of Recife. As a differential, we
do not study only one park, but all urban parks recognized by the city
and which had specific equipment for physical activities. We used
questionnaires applied to users of all parks to characterize them and we
used the contingent valuation method to estimate the monetary value.
Among other contributions, the results demonstrate that the parks in
Recife have a greater coverage than usual and that users with lower
income groups value these areas more than users with lower income.

Separate text sections with \cite{Mislevy06Cog}.

\hypertarget{sec:1}{%
\section{Methodology}\label{sec:1}}

\hypertarget{sec:2}{%
\subsection{Study Area}\label{sec:2}}

The place of study was the Municipality of Recife, capital of the state
of Pernambuco, in the Northeast region of Brazil. The municipality has
10 public parks and countless squares spread over the 94 neighborhoods
that compose it. The area covered by the squares parks is equivalent to
8.2\% of the municipal territory. These 10 areas have structures known
as city gyms, spaces with equipment and classes of different modalities,
in addition to other leisure equipment (Figure 1).

\textbf{inserir figura do mapa}

\hypertarget{survey-methods-and-questionnaire-design}{%
\subsection{Survey methods and questionnaire
design}\label{survey-methods-and-questionnaire-design}}

We conducted the survey between December 2018 and March 2019 among the
local residents of the city of Recife that uses at least one of the 9
parks of the city. The survey method was face-to-face personal
interviews by means of a structured questionnaire and was applied to
1281 questionnaires. All procedures were according to the rules of
resolution 510/1617 (BRASIL, 2016), the opinion survey format was
prepared and did not request any identification.

We gathered data among the Recife users population's universe,
configuring a non-probabilistic sample with a convenience bias. We
recognize the sampling scheme has not achieved a representative sample
of the city's households. All the analyses and estimates were performed
with the R Studio 4.0 software. The dataset is available in csv format
at a Github repository
{[}\url{https://github.com/cccneto/valuation_urbanParks}{]}.

The current survey was tested through a pilot study/previous survey in
one of the city parks (CITAR ARTIGO DO ARACA). We have constructed the
bid vector for the dichotomous choice questions based on the analysis of
the WTP responses in the previous survey. The questionnaire's structure
was based on the NOAA panel recommendations for CVM studies (Arrow et
al., 1993). The questionnaire consisted of a set of 17 questions.

In the first part, we surveyed respondents' socio-economic and household
characteristics from age at least 18 years. In the second part we asked
about: a) the visit purpose to the park, b) the frequency visits to the
park, c) the main criteria determining the decision to visit a park, d)
their perceptions about the park characteristics (e.g.~infrastructure,
maintenance, size, security), finally, e) their perception about the
presence of ecosystem services in the park.

The final part we described the hypothetical scenario and the valuation
questions. In this scenario, we presented a change in the park's vegetal
cover to the respondents, focusing on the transformation of the
environmental quality of the area. It is important to say the
interviewer did a relevant visual presentation of the changes in the
scenario presented (i.e.~images before and after the park development).

In the approach to respondents on the park, we asked residents if they
wanted to participate in this research (i.e., if they agree to respond
to the survey questions). After the two initial parts of the
questionnaire, we asked how much they would be willing to pay for the
changes presented to them. The interviewer explained their answers would
be useful to the decision-making and planning process. About the WTP
question, we adopted a close-ended format to better approximate real
market transactions (i.e., take-it or leave-it decisions). We have have
adopted the a double-bounded (DB) dichotomous choice format following
the recommendations of the NOAA panel (Arrow et al., 1993) and Hanemann
(1984). The procedures presents to respondents an initial bid value,
randomly selected from a set of 70 bid levels - R\$ 1 to R\$ 70. If the
response was ``yes'' a follow-up question with a higher bid was asked,
while a ``no'' response led to a lower bid level. The value amount would
be annually collected by the municipal authority and would be
exclusively devoted to cover the development costs of the park.

\hypertarget{theoretical-model}{%
\subsection{Theoretical Model}\label{theoretical-model}}

According to Groothuis and Whitehead (2002), econometric models of
dichotomous choice have been an instrument widely used to address issues
related to contingent valuation. O Modelo de utilidade randômica fornece
as bases teóricas para a análise de Métodos de Valoração Contingente.In
this model an individual could choose to pay a donation fee for the
conservation of the services provided by the studied area if the
following conditions are met (Hanemann, 1984):

\begin{align}
u(y, X) = u(y-t, q, X)
\end{align}

\begin{align}
u(y, X) = u(y_{j}X_{j}) + \epsilon_{0j})
\end{align}

\begin{align}
v(1, y-t; X) + \epsilon_{1} \ge v(0, y, X)
\end{align}

Where \(u\) is the respondent's utility function, \(v\) is the indirect
utility function, \(1\) represents the donation payment and \(0\)
represents the non-payment, \(y\) is the respondent's individual income,
the amount of the bid made to the respondent, \(X\) represents other
socioeconomic characteristics that affect the respondent's preferences.
The difference between the utilities \(\Delta v\) determines the payment
or not of the donation:

\begin{align}
\Delta v= (1, y-t;X) -v(0, y, X) + \epsilon_{1} + \epsilon_{0}  
\end{align}

The MVC dichotomous choice format requires a qualitative choice model.
The use of a linear distribution of the WTP and a Bivariate Probit Model
(BPM), was developed based on the model by Cameron \& Quiggin (1994). It
is assumed that the error of the second dichotomous question is
correlated with the error of the first question. For this reason, we
follow Alberini's (1995) recommendation for the choice of bivariate
dichotomous models, because if the coefficient correlation,
\(\rho \ne 1\), it is clear that, in general, the second WTP does not
perfectly match the first and can be interpreted as a revised version of
the amount of the first WTP. If WTP values are independently determined,
then \(\rho = 0\). For all other values of the correlation coefficient,
the interval \(0 < \rho < 1\) is valid, which implies that the
correlation between the two WTP values is less than perfect.

Considering these aspects, the modeling of the data generated by the
questions in the double limit dichotomous choice format was achieved by
the following formulation:

\hypertarget{linear-model}{%
\paragraph{Linear Model}\label{linear-model}}

\[
\begin{aligned}
\Delta Y_{i}(yes|no)= \alpha_{0} + \alpha_{1}Age + \alpha_{2}D_{1i} + \alpha_{3}D_{2i} \\ 
+ \alpha_{4}D_{3i} + \alpha_{5}D_{4i} + \alpha_{6}Tempo_{i} + \beta_{1}Bid_{12i} + \epsilon_{i}     
\end{aligned}
\]

\(Y_i\) is the dependent variable and reports the respondent's answer
(\(yes = 1\) or \(no = 0\)) to the \(Bid\), \(Age_i\) is the age of the
respondent, \(D_{1i}\) is the dummie variable for the \(Sex\) of the
respondent (man = \(1\), woman = \(0\)), \(D_{2i}\) is a dummie for the
respondent's education (complete higher education = \(1\)), \(D_{3i}\)
is a dummie for respondent assessing regarding the temperature in the
park (good / excellent = \(1\)), \(D_{4i}\) is a dummie for assessing
the respondent regarding of the park's infrastructure (good / great =
\(1\)), \(Bid_i\) are the variables for the values drawn as bids to
respondents.

The \(WTP_{ij}\) component represents the respondent's \(j-th\)
willingness to pay and \(i = 1\), 2 denotes the first and second
questions, respectively.

\begin{align}
WTP_{ij} = X'_{ij}\beta_{i} + \epsilon_{ij}
\end{align}

The \(WTP\) depends on a systematic component given by the observed
characteristics of the interviewee \((X'_{ij}\beta_{i})\), as well as a
random random component \((ij \sim N(0, \sigma^2))\).

\begin{align}
Pr(yes, no) = Pr(WTP_{1j} \ge t^1, WTP_{2j} < t^2)
\end{align}

\begin{align}
Pr(yes, no) = Pr(X'_{1}\beta_{1} + \epsilon_{1j} \ge t^1, X'_{2}\beta_{2} + \epsilon_{2j} < t^2)
\end{align}

Since the other sequence of possible responses can be constructed in an
analogous way, which allows building the likelihood function:

\[
\begin{aligned}
L_{j} (\mu | t) = Pr(X'_{1}\beta_{1} + \epsilon_{1j} \ge t^1, X'_{2}\beta_{2} + \epsilon_{2j} < t^2)^{yn} * \\
Pr(X'_{1}\beta_{1} + \epsilon_{1j} < t^2 , X'_{2}\beta_{2} + \epsilon_{2j} \ge t^2)^{ny} * \\
Pr(X'_{1}\beta_{1} + \epsilon_{1j} > t^1, X'_{2}\beta_{2} + \epsilon_{2j} \ge t^2)^{yy} * \\ 
Pr(X'_{1}\beta_{1} + \epsilon_{1j} < t^1, X'_{2}\beta_{2} + \epsilon_{2j} < t^2)^{nn} 
\end{aligned}
\]

Given a sample of n respondents, we have that the function of
logarithmic probability of the responses to the first and second moves
of the dichotomous choice with double limit is:

\[
\begin{aligned}
Ln(L_{j} (\mu | t)) = yn \ln((X'_{1}\beta_{1} + \epsilon_{1j} \ge t^1, X'_{2}\beta_{2} + \epsilon_{2j} < t^2) * \\
ny Pr(X'_{1}\beta_{1} + \epsilon_{1j} < t^2 , X'_{2}\beta_{2} + \epsilon_{2j} \ge t^2) * \\
yy Pr(X'_{1}\beta_{1} + \epsilon_{1j} > t^1, X'_{2}\beta_{2} + \epsilon_{2j} \ge t^2) * \\
nn Pr(X'_{1}\beta_{1} + \epsilon_{1j} < t^1, X'_{2}\beta_{2} + \epsilon_{2j} < t^2)) 
\end{aligned}
\]

Once the regression is estimated, the estimated WTP is calculated as:

\begin{align}
\widehat{WTP} = \frac{\hat{\alpha}\overline{X_{i}})}{\hat{\beta}} 
\end{align}

\hypertarget{sec:3}{%
\section{Results}\label{sec:3}}

After processing the data, 1144 questionnaires were used. Considering
the estimated population of 1,653,461 inhabitants of the city, and the
distribution between age groups, income, sex and place of residence, the
data are representative for the city. Among the interviewees, 51.68\%
were female; 48.32 male. The vast majority of respondents 92.82\% were
young adults (\textless65 years), living in 94 neighborhoods in the city
(Table 1).

Regarding the profile of respondents, \(45.9%
\) of the respondents are female, the average age is about 41.3 years
(sd = 14.8). The average monthly income is \(R\$ 1.800\), and the median
income was estimated to be between \(R\$ 41\) and \(R\$ 61.200\). Only
\(25.9%
\) hold a degree from a university or from a technological educational
institute.About the use of dummy variables, we focused on respondents
assessing their self-perception on the temperature in the park, and
regarding the park's infrastructure. About \(31.8%
\) beliefs that the park has a nice temperature because of the natural
cover to the park. And \(55.4%
\) evaluated the "infrastructure as good/great. These variables were
included in the analysis because it was expected to affect positively
the probability to participate in the project, as well the respondent's
WTP.(ver tabela x)

The results from the BP regression for the general sample are shown in
Table x. The estimations for Eq. (1) are shown in the upper part of the
Table, while the estimation for Eq. (2) in the lower part. The
probability that WTP be equal or higher to the two bids amount is
influenced by the respondent's own characteristics and by a series of
independent variables that reflect her/his preferences for the ES being
valued. The coefficient on the bid was negative and significant in both
equations, which indicates as the price increased the probability of a
positive answer to the WTP question decreased.

The results for the general sample indicate that WTP from the first
equation (Eq 1) was influenced positively by the level of education, and
by the respondent's perceptions about infrastructure and temperature of
the park, while was negatively affected by the respondent's age. The
variable gender wasn't statistically significant.

in the second equation, the results for the general sample indicate that
WTP wasn't influenced positively by the age of respondents, but does by
respondent's perceptions about infrastructure. Opposite to the first
equation, the level of education and the respondent's perceptions about
the temperature of the park weren't statistically significant.

\begin{verbatim}
## Data Frame Summary  
## base  
## Dimensions: 1144 x 10  
## Duplicates: 25  
## 
## ---------------------------------------------------------------------------
## No   Variable          Stats / Values                 Freqs (% of Valid)   
## ---- ----------------- ------------------------------ ---------------------
## 1    resp1             1. 0                           481 (42.0%)          
##      [factor]          2. 1                           663 (58.0%)          
## 
## 2    resp2             1. 0                           587 (51.3%)          
##      [factor]          2. 1                           557 (48.7%)          
## 
## 3    lance1            Mean (sd) : 29.9 (17.7)        70 distinct values   
##      [integer]         min < med < max:                                    
##                        1 < 28 < 70                                         
##                        IQR (CV) : 29.2 (0.6)                               
## 
## 4    lance2            Mean (sd) : 33.7 (19.5)        71 distinct values   
##      [integer]         min < med < max:                                    
##                        1 < 32 < 75                                         
##                        IQR (CV) : 34 (0.6)                                 
## 
## 5    idade             Mean (sd) : 41.3 (14.8)        68 distinct values   
##      [integer]         min < med < max:                                    
##                        17 < 39 < 96                                        
##                        IQR (CV) : 21 (0.4)                                 
## 
## 6    sexo              Min  : 0                       0 : 525 (45.9%)      
##      [integer]         Mean : 0.5                     1 : 619 (54.1%)      
##                        Max  : 1                                            
## 
## 7    escolar           Min  : 0                       0 : 848 (74.1%)      
##      [integer]         Mean : 0.3                     1 : 296 (25.9%)      
##                        Max  : 1                                            
## 
## 8    infraestrutura    Min  : 0                       0 : 510 (44.6%)      
##      [integer]         Mean : 0.6                     1 : 634 (55.4%)      
##                        Max  : 1                                            
## 
## 9    temperatura       Min  : 0                       0 : 780 (68.2%)      
##      [integer]         Mean : 0.3                     1 : 364 (31.8%)      
##                        Max  : 1                                            
## 
## 10   renda             Mean (sd) : 3091.1 (3798.1)    131 distinct values  
##      [integer]         min < med < max:                                    
##                        41 < 1800 < 62000                                   
##                        IQR (CV) : 3002 (1.2)                               
## ---------------------------------------------------------------------------
\end{verbatim}

\textbf{inserir tabela com proporcao das respostas y e n's}

\hypertarget{sec:4}{%
\section{Discussion}\label{sec:4}}

\hypertarget{sec:5}{%
\section{Conclusions}\label{sec:5}}

\hypertarget{sec:6}{%
\section{Policy implications}\label{sec:6}}


\bibliographystyle{spphys}
\bibliography{bibliography.bib}

\end{document}
